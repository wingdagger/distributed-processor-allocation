ABSTRACT

A processor allocation algorithm is a method to determine the best processor
within a distributed system on which to run a process.  For example, assume
a machine $X$ with one processor and a high load average.  The user of
machine $X$ creates a new process.  However, since its load average is so
high, machine $X$ decides to offload the process to another machine $Y$.
Hence, a processor allocation algorithm is invoked to determine the best
processor $Y$.  The goal of a processor allocation algorithm is to do this
automatically, completely transparent to the user.

An alteration of the aforementioned approach is to allow processes to
migrate dynamically even after they have started executing.  This is
achieved via preemptive scheduling with the use of checkpointing (saving and
transmitting process state), whereas the former approach is achieved via
non-preemptive scheduling (process runs to completion on the machine where
it is started).

My research looks at three different processor allocation algorithms, one
centralized and two distributed.  All three algorithms were implemented
using kali-scheme, a distributed version of scheme48 \cite{kali}.  Three
environments were used to test the algorithms: simulation, quasi-simulation,
and full implementation.  Test cases highlighting performance, scalability,
and fault tolerance were run to compare and contrast the algorithms.

