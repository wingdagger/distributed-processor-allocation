\documentclass{slides}
%\usepackage{times}
\usepackage{xspace}
\usepackage{graphics}
\usepackage{here}
\usepackage{bar}
\usepackage{shadow}
\usepackage{boxedminipage}
\usepackage{tabularx}


\newcommand{\hopcountbar}[3]{
\begin{center}
\textbf{#1}
\end{center}

\begin{barenv}
\setdepth{10} % 3-D effect
\setstretch{1.4} % stretch y-dimension
\setnumberpos{up} % number above bars
\setxaxis{1}{7}{1}\setxname{Hop Count}
\setyaxis{0}{50}{10}\setyname{No.}
#2
\end{barenv}

\vspace{1em}
\line(1,0){50} \\
#3
}


\begin{document}

\title{Process Creation Request Hop Counts}



\hopcountbar{Hop Counts for Q-learning Algorithm}
{\bar{0}{1}
\bar{28}{2}
\bar{7}{3}
\bar{3}{4}
\bar{2}{5}
\bar{1}{6}
\bar{9}{7}}
{Q-learning algorithm in quasi-simulation environment.  5 processors, 50
processes of mixed process types.  Average of 3 iterations}



\hopcountbar{Hop Counts for Centralized Algorithm}
{\bar{0}{1}
\bar{0}{2}
\bar{50}{3}}
{Centralized algorithm in quasi-simulation environment.  5 processors, 50
processes of mixed process types}



\hopcountbar{Hop Counts for Token-based Algorithm}
{\bar{0}{1}
\bar{50}{2}}
{Token-based algorithm in quasi-simulation environment.  5 processors, 50
processes of mixed process types}




\end{document}
