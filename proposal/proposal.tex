\documentclass{article}
\usepackage{fullpage}
\usepackage{xspace}
\usepackage{educ-homework}

\newcommand{\UNIX}{\textsc{Unix}\xspace}

\setcounter{secnumdepth}{0}
\name{Joshua S. Allen}


\begin{document}

\LARGE
\noindent \textbf{Masters Thesis Proposal:} \\
\noindent \textbf{Distributed Processor Allocation Algorithms} \\
\Large
\noindent {by Joshua S. Allen}

\normalsize
\vspace{1em}

\section{Introduction}

% Say something about distributed systems??

A processor allocation algorithm is a way of determining the best processor
within a distributed system on which to run a process.  Say, for example,
that machine $A$ has only one processor and its load average is very high.
The user of machine $A$ now wants to run a new process.  However, since its
load average is so high, machine $A$ decides to give the process to another
machine.  Hence, a processor allocation algorithm is invoked to determine
the best processor to which to give the process.  An alteration of this
approach is to allow processes to migrate dynamically even after they have
started executing.

Other than automatically offloading or migrating processes within a
distributed, another use of a processor allocation algorithms could be in a
distributed implementation of the Unix utility \texttt{top}.  In the
business world, distributed processor allocation algorithms could be used to
assign tasks of a project to employees.



\section{Goal} 

The goal of this masters thesis is to develop fair processor allocation in
distributed systems using a distributed algorithm.  Three algorithms will be
designed and implemented using kali-scheme, a distributed implementation of
scheme48.  Various information will be logged during testing of the system.
At the end of testing, some of the logged information will be plotted for
statistical analysis, and a comparison of the algorithms will be given.

\section{Criteria}

In general, distributed systems can be compared using the following
criteria \cite{Tanenbaum}.  Distributed processor allocation is no
exception.


\begin{itemize}
	
  \item Transparency

	The distributed system should provide services such that the user is
	unaware of where within the system that the services take place.
	The system should function, at least from the user's perspective, as
	a single unit. 


  \item Flexibility

	The distributed system should be designed in such a way that it is
	easily changeable and customizable.

  \item Reliability

	The distributed system should provide consistency and should
	function even in the presence of failures when possible.  In other
	words the distributed system should be fault tolerant.  When
	possible, failures should go unnoticed by the user.

  \item Performance

	The distributed system should provide a reasonable and consistent
	response time to users and in general the system should provide a
	level of performance not less than that of a stand-alone machine.
	In other words, the distributed system should not penalize
	performance, and when possible it should enhance it.

  \item Scalability

	The design of a distributed system should be such that it functions
	appropriately and efficiently no matter how many machines comprise
	it.  

\end{itemize}


Given a collection of processors, each of which can receive external job
requests, fair processor allocation should be performed for newly incoming
jobs.  In the case of dynamic process migration, fair processor allocation
should occur during the lifetime of processes.  The unfairness at time $t$
can be measured by the maximum difference in jobs between two processors.
The performance of an allocation method over a time period can be measured
by the average unfairness over that time period.

However, load alone is not enough to determine the best processor on which
to offload a process.  From a user point of view, a distributed system
should offer better than or equal to performance of a stand-alone machine.
Hence, criteria such as time delay between process creation and initial
execution is important.  Items that would effect this include hop count of
messages in the network, whether a binary needs to be migrated across the
network, etc. 

Another important variable to check when offloading processes is the amount
of free memory.  If a machine does not have enough free memory to run a new
process then it would be pointless to try running it there.  Additionally,
if a machine does have enough free memory but very little of it, then
performance might be effected due to swapping.  Hence, many factors other
than load need to be taken into account when determining the best machine on
which to offload a process and when comparing the performance of processor
allocation algorithms.


Three measurable statistics come to mind when measuring the
\emph{performance} of a processor allocation algorithm:

\begin{itemize}
	\item Maximizing CPU utilization (comparison of load averages 
	over time)
        \item Minimizing response time (the time elapse from process
        creation to process execution)
        \item Minimizing total elapsed time a process requires to finish
        (the time elapse from process creation to process termination).
\end{itemize}

\emph{Scalability} can be measured by comparing the results of performance
measurements of the algorithms running variable sizes of distributed
systems.

\emph{Fault Tolerance} can be determined by introducing failures into the
distributed and accessing whether the system continues to perform and, if
so, if it performs at the same performance levels.

\emph{Flexibility} is not something that can easily be measured and hence a
descriptive explanation of the features that make an algorithm flexible will
need to be discussed, such a modularity, ease of use, etc.

\emph{Transparency} also is difficult to measure, so again a descriptive
explanation of why and how an algorithm is transparent will be given.


%\section{Background}


\section{Approach}

\subsection{Algorithms}

\subsubsection{Token based Algorithm}

At startup, a logical ring is constructed, where each node in the ring
represents a processor in the distributed system.  For test purposes the
coordinator will be the machine on which the user is interacting.  However,
the coordinator can be elected using an election algorithm.  A single packet
(the token) is then transmitted with an entry for each processor in the
system.  The entry will contain the following information: load and amount
of free memory.

Each time a machine receives the token, it does the following things:

\begin{enumerate}
  \item The machine updates the token's usage table with its current load
  and amount of free memory.

  \item If the machine wants to offload a process, it picks the machine with
  the lowest load that has enough free memory, and migrates the
  process to that machine.  

  \item The machine forwards the token to the next machine in the logical
  ring. 
\end{enumerate}

\subsubsection{Randomized Algorithm}

Each machine maintains a table comprised of values reflecting the last known
load on each machine, initially all values are zero (neutral).  When a new
job is introduced to a processor, that processor allocates the job to some
processor in the system, perhaps itself.  This choice is based on the table
most of the time and made randomly the rest of the time (parameter $p$).

The table is updated in the following way: When processor $A$ chooses to
give a job to processor $B$, $A$ sends a request to $B$, which includes
$A's$ current load.  $B$ updates its table using this information and
decides whether to accept the job or forward the request to another machine.
Ultimately, some machine $C$ accepts the request and obtains the job from
$A$.  The only way a job could bounce around the system indefinitely is for
new jobs to be introduced to the system faster than the system can send
messages back and forth between machines.  This is is highly unlikely, thus
practically there is no indefinite postponement of jobs, although
theoretically it is possible.  Each machine ($A$ through $C$) updates its
table using load information passed to it in the request message, which is
appended to the request as it is forwarded to each machine.  A naive update
rule is simply to replace old data with new data.  However, a better
approach might be to use weighted average of the old and new data, perhaps
incorporating the age of the old data.


\subsubsection{Centralized Algorithm}

For comparison reasons, a centralized version of the algorithm will be
implemented in addition to the distributed ones.  In the centralized
algorithm, all process creation requests go through a single machine,
although they can still originate anywhere.  When a process is offloaded to
a machine by the coordinator, that machine must return its load average to
the coordinator.  The same is true when the process finishes.  Hence, the
coordinator always knows (within time $\epsilon$, defined to be the amount of
time for a client to respond to the coordinator) the load average on every
machine in the distributed system.



\section{Implementation}

The three algorithms will be implemented using kali-scheme, a distributed
implementation of scheme48 \cite{kali}.  Processors will be simulated
processors using threads in a single \UNIX process, several \UNIX processes
on the same physical machine, and several \UNIX processes on multiple
physical machines.

Initially, threads will be used to simulate processors and then several
\UNIX processes on multiple physical machines in order to perform more
realistic tests.  This will be more realistic because network delays and
actual machine usage patterns can be taken into account.


\section{Current Status}

Basic working implementations of all three of the algorithms are complete.
Currently, all processors and processes in the system are simulated using
threads within a single process on a single machine.  Logging and simple
graph generation is implemented so that test results can be easily
compared.  All three algorithms currently have no fault tolerance and assume
static allocation at the beginning of process execution.  Only a few
statistics are currently being logged such as load and amount of free
memory.

A general outline of the paper is prepared with initial background
information in place.  Also available is an initial bibliography.


\section{Plan for Future Work}

All algorithms will be converted to run on multiple physical machines
instead of running as threads on a single machine.  Instead of supplying a
pseudo-load average when creating a process, real load averages will be used
(perhaps thread loads).  More information will be tracked for statistical
analysis, including network delays, hop counts, and the amount of time
between process creation and initial execution.  For the randomized
algorithm, more update algorithms will be developed in order to provide more
optimal results.  More data will be plotted for statistical analysis.  All
of the algorithms will be modified to perform dynamic process migration
(movement of processes during execution among machines).  Finally, the
algorithms will be altered to include fault tolerance.

In addition to programming changes made to the algorithms, more test cases
will be run on the system, varying many variables.  Much analysis will be
done on the results of the test cases in order to provide comparisons of the
algorithms, including statistical analysis and commentary based upon the
work of others and my own thoughts.  

Results from the current implementation of the algorithms, where processors
and processes are fully simulated will be compared to the final results
where each processor in the system represents a physical processor.  If
results differentiate significantly, then an analysis of how the simulation
could be changed will be provided.  Other test cases will be performed in a
quasi-simulated environment where some things are simulated, such as
processes, yet other things are not, such as processors.

Finally, an analysis of efficient ways to offload (static) and migrate
(dynamic) processes in a distributed system will be given.



\bibliography{proposal}
\bibliographystyle{alpha}

\end{document}









